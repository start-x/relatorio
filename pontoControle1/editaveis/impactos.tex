\chapter[Impactos]{Contribuições e prováveis imapctos}

Com a criação do produto Bike-x será possivel criar em academias, ou mesmo na sala de uma casa, um ambiente divertido para realização de exercício físico. Além de se divertir em um ambiente de realidade aumentada, espera-se que o produto seja energeticamente autosuficiente, onde a energia gerada pela própria realização do exercício irá alimentar o produto assim como algum dispositivo móvel que o usuário deseje carregar a sua bateria, tendo assim um viés sustentável.  

O produto trará mais segurança para quem gosta de andar de bicicleta, o usuário não precisará se expor a ambientes perigosos como andar ao lado de carros, ônibus, motos, caminhões, motoristas destraídos e muitas vezes bêbados, possibilitando até mesmo o usuário andar de bicicleta em um inverno rigoroso e com neve. Via esse produto cria-se a expectativa de diminuição do sedentarismo entre as pessoas, reduzindo a lista de fatores complicadores para a realização de atividades física. Um outro beneficio significativo é que ao utilizar o dispositivo será possivel monitorar algumas informações do individuo que está utilizando o produto, em relação a batimento cardíaco, distância percorrida, entre outras.

Apesar dos benefícios do produto mostrados anteriormente devemos levar em conta os possíveis impactos gerados ao utilizar a Bike-x, como desconforto após utilizar o óculos por muito tempo e a existencia de menos bicicletas sendo utilizadas nas ciclovias.
