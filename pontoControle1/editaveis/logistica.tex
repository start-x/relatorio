
\chapter*[Logística]{Logística}
\addcontentsline{toc}{chapter}{Logística}

Para a realização do projeto de forma eficiente e organizada, dividiu-se o grupo em quatro subgrupos, cada um destes representando uma das engenharias (automotiva, eletrônica, energia e software), de modo que ao final da realização de cada tarefa os outros subgrupos sejam informados.

As decisões importantes a serem tomadas, como a definição do tema do projeto, o cronograma, as divisões e os principais resultados esperados, são feitas por todos os componentes do grupo durante os horários de aula da disciplina, contabilizando seis horas semanais. Além das tomadas de decisões, essas horas são aproveitadas para cada subgrupo se reunir, trabalhar em sua determinada área, apresentar e discutir seus resultados obtidos para os demais subgrupos e, quando necessário, apresentar suas principais dificuldades e questionamentos para os professores da disciplina. Essas horas também serão uteis para que as tarefas em que é necessário mais de um subgrupo para sua realização sejam cumpridas através da reunião entre os mesmos para coletar as informações necessárias e discutir os melhores métodos e soluções para essas tarefas.

Foi estimada uma média de quatro horas semanais de trabalho além das seis horas de aula para cada componente do grupo, a fim de concluir as tarefas e metas propostas para cada um desses. Essas horas são utilizadas em sua maioria para pesquisas, testes, simulações e atualizações do relatório. 

Com o objetivo de aperfeiçoar a integração entre os componentes dos grupos e para que cada um possa acompanhar o andamento do projeto são utilizadas algumas ferramentas virtuais, como redes sociais e plataformas de controle de projeto. Assim, de modo que cada componente e/ou subgrupo possa acompanhar o que os outros estão fazendo no projeto é utilizada a ferramenta Readmine, onde são apresentadas as tarefas, seus andamentos e o responsável por cada uma delas, possibilitando com que os principais problemas e dificuldades sejam detectados. Para o agrupamento dos dados e pesquisas coletadas, além dos testes e resultados gerados e atualizações do relatório, é utilizada a ferramenta Google Docs. A rede social utilizada para proporcionar a comunicação entre todos os integrantes do grupo a fim de resolver problemas, dividir tarefas e marcar encontros é o Facebook.

Com essa maneira de organizar o tempo, as tarefas e as equipes, espera-se que o andamento do projeto seja satisfatório, integrando as engenharias através do trabalho entre os subgrupos de maneira eficiente. Além disso, objetiva-se o melhor aproveitamento possível das horas disponíveis e determinadas para a realização do projeto por todos os componentes, de modo que a divisão de trabalho seja equilibrada ao longo do projeto, o que pode ser observado e analisado através das ferramentas utilizadas para o controle e divisão de tarefas.

