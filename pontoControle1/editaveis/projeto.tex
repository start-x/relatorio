\chapter[Projeto]{Projeto}
O projeto contará com diversas interações com o usuário que tem como objetivo fornecer a melhor sensação de utilizar uma bicicleta em um parque. O projeto irá ser feito de forma modular, para que a constituição final do produto seja feita com base nas necessidades de cada usuário. 

Nesse sentido, o protótipo contará com uma bicicleta e uma base que possa fixar essa bicicleta. Para dar a sensação de que o usuário estará pedalando, utilizaremos um rolo para que a roda traseira possa ser utilizada normalmente. O guidão também será utilizado de forma natural e um sensor será utilizado para definir qual a direção o usuário está virando. 

Para a realidade virtual, iremos fazer a modelagem de um ambiente e utilizando o \textit{Oculus Rift}, iremos fazer uma imersão do usuário em uma área que normalmente seria utilizada para um passeio de bicicleta. O usuário poderá ter acesso a algumas informações referentes ao seu passeio e exercício, tais como: velocidade, batimento cardíaco e outras. 

O \textit{Oculus Rift} representa hoje no mercado a ferramenta que tange a fronteira da imersão em uma realidade virtual e ao mesmo tempo a acessibilidade ao usuário. As primeiras aparições de tecnologias semelhantes de imersão virtual são datadas de 1961 pela Philco\cite{boasoverview}. Atualmente hoje há diversas opções de dispositivos que oferecem uma pervasividade especifica, tais como \textit{Cave Automatic Virtual Environm}(CAVE), \textit{Head-Mounted Displays}(HMDs) e dispositivos de entrada, tais como controles sem fio ou câmeras de rastreamento.

Uma interação que será feita a partir do ambiente com o usuário será em caso de subidas no mesmo. Para dar a sensação de dificuldade que se tem ao pedalar em uma subida, iremos acionar de acordo com a intensidade da subida, os freios da bicicleta nesses momentos. 

Por fim, planejamos entregar um produto auto sustentável, ou seja, será feita a conversão de energia eletromecânica do exercício em energia elétrica. Acoplando um alternador ao rolo da bicicleta, poderemos alimentar parte do sistema ou até mesmo como um todo.

