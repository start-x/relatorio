\chapter*[Resultados]{Resultados}
\addcontentsline{toc}{chapter}{Resultados}

Com a inclusão dos sensores no sistema, espera-se coletar dados que são considerados importantes para as mudanças físicas que irão ocorrer no ambiente virtual, bem como causar sensações no usuário de forma que o mesmo tenha um experiencia semelhante a andar de bicicleta na rua. Os seguintes dados serão coletados: oximetria, velocidade, direção a qual o guidão é movimentado e nível da bateria para o sistema de realimentação. Por meio da oximetria serão obtidos a saturação de oxigênio da hemoglobina e a frequência cardíaca em batimentos por minuto.

Os dados da oximetria e velocidade serão apresentados ao atleta para ele tenha consciência do seu desempenho e para manter os batimentos cardíacos em um nível desejado. O sinal do sensor de direção do guidão fará com que haja uma alteração no ambiente virtual, ou seja, dependendo da angulação do guidão, o usuário terá a sensação de que está fazendo uma curva. Outro dado importante é que quando houver uma subida no percurso do ambiente virtual o usuário terá maior dificuldade ao pedalar, como se realmente estivesse subindo um morro, por exemplo.

Por meio destas características, espera-se simular um ambiente que seja tão próximo quanto possível da realidade de forma a tornar atividades físicas, como o spinning, algo menos monótono já que o atleta não se desloca e apresentar dados que possam melhorar seu rendimento. Outro resultado interessante é realizar a comparação do nível de iteratividade do sistema com uma situação real e analisar como o usuário se comporta em ambos os ambientes. Isso é interessante para atletas de alto nível pois simular o percurso de uma prova e ter conhecimento das reações do corpo naquele ambiente é de fundamental importância para um bom desempenho.

