  
  \def\firstcircle{(0:3.75cm)  circle (3.85cm)}
  \def\secondcircle{(90:3.75cm) circle (3.85cm)}
  \def\thirdcircle{(180:3.75cm) circle (3.85cm)}
  \def\fourdcircle{(270:3.75cm) circle (3.85cm)}

\tikzstyle{sensor}=[draw, fill=blue!20, text width=5em, 
    text centered, minimum height=2.5em,drop shadow]
\tikzstyle{wa} = [sensor, text width=10em, fill=red!20, 
    minimum height=6em, rounded corners, drop shadow]

  \begin{tikzpicture}[->,>=stealth',shorten >=1pt,auto,node distance=3cm,
  thick]

    % The last trick is to cheat and use transparency
    \begin{scope}[shift={(3cm,3cm)}, fill opacity=0.5]
        \fill[red] \firstcircle;
        \fill[green] \secondcircle;
        \fill[blue] \thirdcircle;
        \fill[yellow] \fourdcircle;
  %       \foreach \A in {left,above} {
  %       	\draw \firstcircle node[below] {$A$};
		% 	\draw[thick] (0.2+\A,2.0-\A) -- (0.2+\A,2.01-\A) arc (-70:70:0.5+\A) ;
		% }
        \draw \firstcircle node[below] (a) {$Software$}
        	[clockwise from=0,concept/.append style={level distance=3cm,sibling angle=30}]
			child { node[concept] {Gerenciamento} }
			child {node[concept] {Modelagem} }
			child { node[concept] {Óculos de Realidade Virtual} }
        ;
        \draw \secondcircle node [above] (b) {$Eletrônica$}
        [clockwise from=90,concept/.append style={level distance=3cm,sibling angle=30}]
        	child { node[concept] {Sensores} }
        ;
        \draw \thirdcircle node [below] (c) {$Energia$}
        [clockwise from=180,concept/.append style={level distance=3cm,sibling angle=30}]
        	child { node[concept] {Armazenamento de Energia} }
        	child { node[concept] {Conversão Electromecânica} }
        	child { node[concept] {Eficiência Energética} }
        	child { node[concept] {Distribuição} }
        ;
        \draw \fourdcircle node [below] (d) {$Automotiva$}
        [clockwise from=270,concept/.append style={level distance=3cm,sibling angle=30}]
        	child { node[concept] {Ergonomia($Percentil$)} }
        	child { node[concept] {Analise Estrutural} }
        ;
    \end{scope}


\end{tikzpicture}