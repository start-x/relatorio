\chapter[Interações]{Interações Entre as Áreas}

O projeto Bike-x contará com a interação das engenharias de Software, Energia, Automotiva e Eletrônica. Esta seção visa identificar e exemplificar cada uma 
das interações. 

As engenharias de software e eletrônica vão se unir na área de coletar dados do sistema de forma geral e gerar informações relevantes a partir das mesmas. 
A área de engenharia eletrônica será responsável em fazer que os microcontroladores leiam dados de diversos sensores, tais como: oximetria e potenciômetro para 
definir a direção do guidão.  A engenharia de software irá processar esses dados e gerará informações para interagir com o usuário, informando os valores ou 
agindo no sistema.

As engenharias de energia e software irão interagir utilizando a eletrônica como intermediária. Haverá um sensor que medirá a quantidade de energia gerada 
pelo usuário e essa informação será tratada via software para que o usuário tenha acesso a essa informação de uma maneira agradável dentro do ambiente virtual.

As engenharias automotiva e de software também irão interagir utilizando a eletrônica como mediadora. Uma das interações com o sistema será que quando houver 
uma subida no ambiente virtual, o software irá gerar uma alteração no sistema. Quando isso houver, a bicicleta irá ser freada.

O sensor que mede a quantidade de energia produzida será responsável pela interação das engenharias de energia e eletrônica. Com esse sensor será possível 
mostrar ao usuário o quanto de energia ele conseguiu produzir. 

Durante uma subida no ambiente virtual, o sistema deverá freiar a bicicleta para que o usuário sinta resistência ao pedalar e tenha a impressão de maior 
dificuldade de pedalar e que realmente pense que esteja em uma subida. Essa será a interação de engenharia eletrônica e automotiva.

Por fim, a interação das engenharias de energia e automotiva será feita pela adaptação do alternador no sistema para a geração de energia. Esse alternador 
será adaptado no rolo que usaremos para que o usuário possa pedalar sem se mover e com esse movimento a energia será gerada para alimentar o sistema.

\begin{figure}[h]
  \centering
    
  \def\firstcircle{(0:2.75cm)  circle (2.85cm)}
  \def\secondcircle{(90:2.75cm) circle (2.85cm)}
  \def\thirdcircle{(180:2.75cm) circle (2.85cm)}
  \def\fourdcircle{(270:2.75cm) circle (2.85cm)}

\tikzstyle{sensor}=[draw, fill=blue!20, text width=5em, 
    text centered, minimum height=2.5em,drop shadow]
\tikzstyle{wa} = [sensor, text width=10em, fill=red!20, 
    minimum height=6em, rounded corners, drop shadow]

  \begin{tikzpicture}[->,>=stealth',shorten >=1pt,auto,node distance=5cm,
  thick,scale=0.8]

    % The last trick is to cheat and use transparency
    \begin{scope}[shift={(3cm,3cm)}, fill opacity=0.5]
        \fill[red] \firstcircle;
        \fill[green] \secondcircle;
        \fill[blue] \thirdcircle;
        \fill[yellow] \fourdcircle;

        \draw \firstcircle node[right] (a) {$Software$}
        	[clockwise from=30,level distance=140,sibling angle=43]
			child { node[concept,text width=3cm,align=center] {Gerenciamento} }
			child {node[concept,text width=3cm,align=center] {Modelagem} }
			child { node[concept,text width=3cm,align=center] {Óculos de Realidade Virtual} }
        ;
        \draw \secondcircle node [above] (b) {$Eletrônica$}
        [clockwise from=90,level distance=170,sibling angle=30]
        	child { node[concept,text width=3cm,align=center] {Sensores} }
        ;
        \draw \thirdcircle node [left] (c) {$Energia$}
        [clockwise from=240,level distance=140,sibling angle=45]
        	child { node[concept,text width=3cm,align=center] {Armazenamento de Energia} }
        	child { node[concept,text width=3cm,align=center] {Conversão Electromecânica} }
        	child { node[concept,text width=3cm,align=center] {Eficiência Energética} }
        	child { node[concept,text width=3cm,align=center] {Distribuição} }
        ;
        \draw \fourdcircle node [below] (d) {$Automotiva$}
        [clockwise from=295,level distance=170,sibling angle=40]
        	child { node[concept,text width=3cm,align=center] {Ergonomia} }
        	child { node[concept,text width=3cm,align=center] {Analise Estrutural} }
        ;
    \end{scope}


\end{tikzpicture}
  \caption{Disponibilizações entre áreas}
  \label{intera}
\end{figure}

