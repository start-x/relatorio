\chapter*[Interações]{Interações Entre as Áreas}

O projeto Bike-x contará com a interação das engenharias de Software, Energia, Automotiva e Eletrônica. Esta seção visa identificar e exemplificar cada uma das interações. 

As engenharias de software e eletrônica vão se unir na área de coletar informações do sistema de forma geral e gerar respostas a partir das mesmas. A área de engenharia eletrônica será responsável em fazer que os microcontroladores leiam dados de diversos sensores, tais como: oximetria e potenciômetro para definir a direção do guidão.  A engenharia de software irá coletar essas informações e gerará respostas para interagir com o usuário, informando os valores ou agindo no sistema.

As engenharias de energia e software irão interagir utilizando a eletrônica como intermediária. Haverá um sensor que medirá a quantidade de energia gerada pelo usuário e essa informação será tratada pela engenharia de software para que o usuário tenha acesso a essa informação de forma de pop up no ambiente virtual.

As engenharias de automotiva e software também irão interagir utilizando a eletrônica como mediadora. Uma das interações com o sistema será que quando houver uma subida no ambiente virtual, o software irá gerar uma alteração no sistema. Quando isso houver, a bicicleta irá ser freada pelo (pegar nome do motor que freia a bicicleta).

O sensor que mede a quantidade de energia produzida será responsável pela interação das engenharias de energia e eletrônica. Com esse sensor será possível mostrar ao usuário quantos (pegar unidade de medida) de energia ele conseguiu produzir. 

Durante uma subida no ambiente virtual, o sistema deverá freiar a bicicleta para que o usuário sinta resistência ao pedalar e tenha a impressão de maior dificuldade de pedalar e que realmente pense que esteja em uma subida. Essa será a interação de engenharia eletrônica e automotiva.

Por fim, a interação das engenharias de energia e automotiva será feita pela adaptação do alternador no sistema para a geração de energia. Esse alternador será adaptado no rolo que usaremos para que o usuário possa pedalar sem se mover e com esse movimento a energia será gerada para alimentar o sistema.

