\chapter[Introdução]{Introdução}



\section[Contexto]{Contexto}

Em 1999 existia no Brasil cerca de 1,7 milhão de usuários de lente de contato. Este valor correspondia a menos de 2\% da população brasileira, sendo que nos EUA esse valor chegava a 15\% do total da sua população \cite{ghanem}.

O número de usuários de lentes de contato cresceu bastante nos últimos tempos, isto se deve ao fato de que ocorreu uma modernização do processo de fabricação do produto. Além disso, a indicação do uso das lentes para pacientes com dificuldades na visão também cresceu, sendo 85 a 90\% por razões óticas \cite{lui}. 

Estima-se que hoje o Brasil tenha cerca de 2 milhões de usuários de lentes de contato. Uma pesquisa recente da UNICAMP demonstra que 50\% dos adultos do país têm idade entre 18 e 39 anos (que é a faixa etária predominante entre os usuários de lentes de contato no Brasil e que precisa de correção visual). Há, portanto, um potencial de usuários enorme de 54 milhões de pessoas \cite{estimativas}.

Se considerarmos que o Brasil siga as tendências internacionais demonstradas em países como E.U.A e Japão, onde o percentual da população que usa lentes de contato é de 8\%, este mercado pode pular para 14.4 milhoes, ou seja, dar um salto de 7 vezes.

Ainda, segundo pesquisas do IBGE, 100 milhões de brasileiros apresentam alguma forma de deficiência visual, mas somente 16\% usam lentes. Outra pesquisa mostra que, dos 84 milhões de usuários exclusivos de óculos, 58\% estão propensos ao uso de lentes de contato \cite{estimativas}.

Existe ainda uma dificuldade inerente ao manuseio das lentes de contato por parte dos usuários. Isto está relacionado tanto ao momento de se colocar e retirar a lente de contato dos olhos quanto ao momento de higienização da lente. Sendo assim, para usuários com algum tipo de dificuldade motora, pacientes com doenças crônicas de tremor ou mesmo para aqueles que não conseguem ver a lente no momento de sua colocação (devido a doenças oculares), esta dificuldade é crítica. Além disso, grande parte das pessoas não usam lentes de contato por não conseguirem realizar o manuseio delas corretamente. 


\section[Problemas]{Problemas}

Por meio da entrevista realizada com o optometrista Leandro Rosa (Anexo A - Entrevista por Pauta) e por meio de pesquisas, pudemos identificar alguns problemas que os usuários de lentes de contato enfrentam em sua utilização. Estes problemas se enquadram em duas categorias principais: dificuldade de utilização e higienização ou limpeza da lente não adequada. 

\textbf{Dificuldade de Utilização}

Segundo o optometrista, é grande o número de pessoas que não utilizam lentes de contato por não conseguirem realizar o processo de inserção e remoção da lente, dentre os fatores que dificultam a sua utilização estão:

\begin{enumerate}
\item Unhas grandes, principalmente entre o público feminino, causam machucados ao redor dos olhos dos usuários ou danificam as lentes de contato.
\item Dificuldade de manter os olhos abertos. Muitas pessoas não conseguem manter as pálpebras abertas quando percebem a aproximação da lente de contato no olho.
\item Hipermetropia. Os usuários que possuem este tipo de doença oftalmológica não conseguem visualizar objetivos próximos aos olhos, o que dificulta o processo de inserção e remoção da lente.
\item Mal de Parkinson ou outras doenças motoras. Os usuários que possuem dificuldades motoras não conseguem manusear lentes de contato adequadamente, inviabilizando o processo de inserção e remoção das mesmas.
\item Crianças. As crianças não conseguem realizar o processo de remoção e inserção de lentes de contato sem o auxílio de um adulto. 
\end{enumerate}

\textbf{Limpeza Inadequada}

Segundo o optometrista, muitos usuários realizam uma higienização inadequada após o uso das lentes de contato, o que pode ocasionar irritação nos olhos, conjuntivite ou até mesmo doenças mais graves como a cegueira por causa do acúmulo de micro-organismos.

\section[Justificativa]{Justificativa}

Com os problemas identificados anteriormente, é possível observar que o desenvolvimento de um produto como o proposto nesse projeto será útil para auxiliar diversas pessoas no processo de inserção, remoção e limpeza das lentes de contato. O Posicionador de Lente poderá ser utilizado tanto por pessoas que utilizam lente por motivo estético quanto por pessoas que precisam utilizar lentes de contato devido a problemas de saúde. Além disso, a perspectiva de mercado de lente de contatos no Brasil é de crescimento e não foram encontrados produtos no mercado que tenham o mesmo objetivo do produto desse projeto. Os produtos patenteados encontrados (Fig. \ref{fig01} e Fig. \ref{fig02}) são apenas para o processo de inserção e remoção das lentes de contato, não realiza processo de limpeza, e tal processo é feito manualmente, o que não atende o público com dificuldade motora, com hipermetropia e crianças, por exemplo. Assim, o projeto do Posicionador de Lente justifica-se não só por ter um grande público com dificuldades de utilização de lentes de contato a ser atingido, impacto econômico, como também pelo impacto social, possível redução de doenças causadas pelas limpeza inadequada da lente e possibilidade de parcerias com programas do governo como o Projeto Olhar Brasil.

\section[Objetivos]{Objetivos}

O objetivo geral deste trabalho é projetar e construir um produto capaz de realizar o processo de inserção e remoção de lentes de contato em seus usuários e capaz de realizar um processo adequado de limpeza das lentes. Os objetivos específicos desse projeto são:

\begin{itemize}
\item Integrar as quatros áreas das engenharias da Faculdade UnB Gama (Engenharia de Software, Engenharia Automotiva, Engenharia Eletrônica e Engenharia de Energia) de forma que os integrantes da equipe possam entregar um produto funcional ao final do semestre letivo.
\item Planejar, Monitorar e Controlar Projeto.
\item Desenvolver intrumentação e sensoriamento.
\item Desenvolver sistema de processamento de imagem.
\item Desenvolver atuadores e sistema de controle.
\item Conceituar, dimensionar e realizar análise estrutural do produto.
\item Projetar e construir recipiente de armazenamento e limpeza.
\item Desenvolver sistema de sucção.
\item Definir alimentação do sistema.
\item  Integrar os componentes ou módulos do projeto e realizar testes do produto.
\end{itemize}

\section[Visão Geral]{Visão Geral}

O Posicionador de Lente é um produto que irá realizar o processo de inserção e remoção de lentes de contato em seus usuários, assim como o processo de limpeza das lentes. Ele será um produto para uso pessoal, assim como os óculos de grau, ou seja, será único para cada pessoa. O produto será calibrado manualmente, inicialmente por um optometrista,  para se ajustar adequadamente a posição dos olhos do usuário. Uma parte mecânica do produto irá movimentar-se sobre um trilho para recolher a lente do recipiente de armazenamento e esse mesmo componente irá inserir a lente no olho do usuário, repetindo o mesmo processo para o segundo olho. Uma pinça de ponta emborrachada será acionada para realizar a abertura dos olhos do usuário. Por meio de sensoriamento, será identificado o toque do instrumento com o olho do usuário para que ele insira e remova a lente de forma confortável e segura. Como componente de segurança adicional, um processamento de imagem será feito para verificar se os olhos do usuário estão abertos durante todo o processo, pois o usuário pode exercer uma força para fechar os olhos ou recuar a cabeça. Após a remoção das lentes, o produto realiza um processo adequado de limpeza e armazenamento das mesmas para o próximo uso.

\subsection[Público Alvo]{Público Alvo}

Como apresentado na subseção Problemas, é diverso o tipo de público que possui problemas em utilizar lentes de contato. Com isso, este projeto desenvolverá um produto que tem como público alvo diversos segmentos, como exemplos: crianças, mulheres com unhas grandes, hipermetrópicos, idosos ou pessoas com dificuldades motoras. Além disso, com a perspectiva de crescimento do mercado de lentes de contato no Brasil, o número de pessoas que terão dificuldades em utilizar lentes de contato também poderá crescer, o que aumenta também o público alvo do produto desse projeto, englobando qualquer pessoa que tenha dificuldade na utilização das lentes de contato.