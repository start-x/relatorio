\begin{resumo}
 
A utilização incorreta de lentes de contato e a limpeza inadequada das mesmas
pode causar danos à saúde das pessoas, como o agravamento de doenças visuais,
irritações nos olhos, conjuntivite ou até mesmo cegueira. Muitas pessoas que deveriam 
ou gostariam de utilizar lentes de contato não as utilizam por dificuldade de manuseio,
inserção e remoção das lentes. Não existe hoje no mercado um produto capaz de satisfazer
as necessidades dessas pessoas: inserção, remoção e limpeza de lentes com mínima interferência
manual. Assim, propõe-se o desenvolvimento de um posicionador de lentes de contato capaz de inserir, remover
e limpar adequadamente as lentes de contato de forma confiável, segura, acessível e salubre. O presente
relatório, destina-se a documentar, descrever e discutir os resultados obtidos nas fases iniciais de
desenvolvimento do projeto, mostrando a metodologia aplicada, as modificações feitas no conceito 
do produto, testes elaborados e resultados parciais obtidos. Também é descrito os planos para o horizonte do
projeto, ressaltando as próximas atividades a serem desenvolvidas afim de concluir o protótipo funcional.

 \vspace{\onelineskip}
    
 \noindent
 \textbf{Palavras-chaves}: lentes. contato. posicionador.
\end{resumo}
