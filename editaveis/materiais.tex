\chapter[Materiais]{Materiais}

Para a execução do presente projeto foram levantados os materiais que serão requisitados para a correta condução deste trabalho. Assim, são informados a seguir 
os materiais necessários, bem como o papel executado por cada material.

\section{Sistema Mecânico/Suporte de apoio}

\begin{itemize}
\item Bicicleta – será o meio pelo qual o usuário do produto realizará atividade física, e a partir dessa atividade, serão gerados dados que servirão de 
  entrada para os sensores. Também será a partir dessa atividade que ocorrerá o acionamento mecânico do gerador.
\item Cavalete – servirá de apoio para a roda traseira da bicicleta, de maneira que esta não entre em contato com o solo.
\end{itemize}

\section{Circuito Elétrico}

Em relação ao circuito elétrico que será responsável para a conversão eletromecânica de energia, distribuição dessa entre os diversos elementos da malha, 
bem como os dispositivos de proteção do mesmo, serão necessários os seguintes materiais:

\begin{itemize}
\item Gerador – será o elemento do circuito que irá realizar a conversão eletromecânica da energia oriunda do sistema escopo deste projeto. Para isso, será 
  utilizado um gerador que utiliza o princípio da indução eletromagnética; especificamente, utilizar-se-á um alternador automotivo.
\item Condutores – terão a responsabilidade de permitir o trânsito de corrente elétrica entre os diversos dispositivos do circuito. Convém informar que as 
  bitolas dos condutores serão dimensionadas visando atender as características elétricas das cargas que serão alimentadas.
\item Dispositivos de proteção – responderão pela segurança do circuito, isto é, cuidarão para que o circuito responda de maneira adequada quando submetido a 
  possíveis distúrbios de ordem elétrica. Assim, permitirão segurança pessoal, integridade dos dispositivos do circuito, bem como isolar o sistema em caso de 
  falta. Para o nosso projeto, propõe-se o uso de fusíveis.
\item Direcionadores de corrente – serão utilizados com o intuito de garantir a correta polarização do circuito em comento. Nesse sentido, serão selecionados 
  diodos que atendam aos requisitos do nosso circuito.
\item Bateria – será o elemento do circuito que armazenará parcela da energia eletromecânica convertida. Desse modo, em caso de falta, ou após a paralisação 
  de funcionamento do gerador, a bateria será responsável pela continuidade da alimentação elétrica das cargas do circuito, garantido certo nível de confiança 
  de fornecimento de energia para as cargas existentes.
\item Multímetro – responsável pela medição das grandezas de ordem elétrica do circuito, tais como tensão e corrente elétrica.
\item Inversor de tensão – será o dispositivo do circuito responsável pela mudança de corrente contínua para corrente alternada.
\item Regulador de tensão – este dispositivo será utilizado para controlar as possíveis flutuações de tensão que poderão existir no circuito, garantindo assim, 
  a correta potência para cada carga.
\item Interruptor – terá a responsabilidade de seccionar temporariamente uma parte do circuito, e também desligá-lo, quando não estiver sendo utilizado. 
\end{itemize}

\section{Cargas}

Os materiais relacionados às cargas que serão conectadas ao circuito elétrico pertinente ao projeto em discussão são indicados a seguir:

\begin{itemize}
\item Dispositivo móvel – uma vez que ocorrerá conversão eletromecânica de energia, parcela dessa energia será disponibilizada para alimentar algum
  dispositivo móvel de interesse do usuário.
\item Sensores – serão alimentados eletricamente sensores que detectarão dados de interesse para o projeto, sendo esses dados repassados ao usuário.
\item Potenciômetro - para detectar virada do guidão.
\item Notebook;
\item Oculus Rift;
\item MSP430 - microcontrolador responsável por coletar os dados dos sensores e transmiti-los para o computador.
\end{itemize}

\section{Software}

Unity 4 – será responsável pela modelagem do ambiente virtual que será exibido ao usuário através do \textit{Oculus Rift}.

