\chapter[Requisitos do Produto]{Requisitos do Produto}

O levantamento de requisitos é o início para toda a atividade de desenvolvimento de um produto, para tanto são utilizadas técnicas de levantamento. Nesse projeto o levantamento de requisitos ocorreu por meio de duas técnicas:

\textbf{1. Entrevista}

A entrevista por pauta foi utilizada para coletar dados úteis para o entendimento do problema, para a definição das necessidades e para a definição de requisitos do projeto. A entrevista foi orientada por uma relação de pontos de interesse explorados pelos entrevistadores ao longo do seu curso e foi registrada em vídeo. O entrevistador fez poucas perguntas diretas e deixou o entrevistado se expressar livremente, e interviu de maneira sutil à medida que se observava o afastamento da pauta. A entrevista foi realizada com um profissional da área de optometria, que foi um dos mentores da ideia do projeto. Essa técnica foi importante para que a equipe entendesse o contexto do projeto e tomasse decisões técnicas importantes. A entrevista por pauta pode ser encontrada no Anexo A - Entrevista por pauta.

\textbf{2. Documentação/Pesquisa}

A técnica de leitura de documentos e pesquisa foi bastante utilizada em todos os módulos do projeto. Essa técnica auxiliou na construção da fundamentação teórica e no entendimento do estado da arte. Com isso, a equipe conseguiu delinear e delimitar o escopo do trabalho.

\textbf{Requisitos Funcionais}

Os requisitos funcionais dizem respeito as principais funcionalidades esperadas do produto a ser construído. São eles:

\begin{itemize}
\item Retirar lente do recipiente de armazenamento.
\item Movimentar pinça para abrir o olho.
\item Verificar durante o uso do produto se os olhos estão abertos.
\item Inserir lente.
\item Retirar lente.
\item Mover lente para o recipiente de armazenamento e limpeza.
\end{itemize}

\textbf{Requisitos Não Funcionais}

Os requisitos não funcionais do sistema dizem respeito às características que o produto deve possuir para ser um produto com qualidade. São eles:

\begin{itemize}
\item Confiabilidade: o produto deve realizar e manter seu funcionamento em circunstâncias de rotina, bem como em circunstâncias hostis e inesperadas.
\item Segurança: o produto não deve submeter os usuários a riscos ou perigo.
\item Acessibilidade: o produto deve poder ser utilizado por pessoas com mobilidade reduzida.
\item Salubridade: o produto não deve afetar a curto, médio ou longo prazo, ao menos de forma potencial, a saúde dos usuários. 
\end{itemize}
