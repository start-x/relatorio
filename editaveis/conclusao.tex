
\chapter[Conclusão]{Conclusão}

No que diz respeito ao projeto,conseguimos integrar de forma eficiente as quatros áreas da engenharia da Faculdade UnB Gama,
de forma que os integrantes da equipe adquiram aptidão para trabalhar em equipes multidisciplinares e que seja produzido algo de valor para
o mercado.
 
No que diz respeito ao produto, considerando as restrições de tempo e complexidade do projeto, desenvolvemos um protótipo do posicionador de lente capaz de cumprir parte dos requisitos funcionais e não funcionais definidos anteriormente e capaz de atender as necessidades do público alvo. O protótipo foi
resultado da integração dos diversos módulos que compõem o produto: instrumentação, controle, sensoriamento,
design, documentação do produto e etc.  

Como apresentado no capítulo anterior, existem melhorias e trabalhos futuros que podem ser realizados para entrega de um protótipo do produto que atenda à todos os requisitos específicados.

Concluímos que o projeto foi de suma importância para o aprendizado de todos os membros da equipe e que devido as limitações de orçamento, de materiais e de tempo conseguimos superá-las e construir um protótipo que atenda aos principais requisitos do Posicionador de Lente.