\chapter[Resultados]{Resultados}

Com a inclusão dos sensores no sistema, espera-se coletar dados que são considerados importantes para as mudanças físicas que irão ocorrer no ambiente virtual, bem como causar sensações no usuário de forma que o mesmo tenha um experiência semelhante a andar de bicicleta na rua. Os seguintes dados serão coletados: velocidade, direção a qual o guidão é movimentado e nível da bateria para o sistema de realimentação.

Os dados da velocidade serão apresentados ao atleta para que ele tenha consciência do seu desempenho. O sinal do sensor de direção do guidão fará com que haja uma alteração no ambiente virtual, ou seja, dependendo da angulação do guidão, o usuário terá a sensação de que está fazendo uma curva. Outro dado importante é que quando houver uma subida no percurso do ambiente virtual o usuário terá maior dificuldade ao pedalar, como se realmente estivesse subindo um morro, por exemplo.

Por meio destas características, espera-se simular um ambiente que seja tão próximo quanto possível da realidade de forma a tornar atividades físicas, como o \textit{spinning}, algo menos monótono já que o atleta não se desloca e apresentar dados que possam melhorar seu rendimento. Outro resultado interessante é realizar a comparação do nível de iteratividade do sistema com uma situação real e analisar como o usuário se comporta em ambos os ambientes. Isso é interessante para atletas de alto nível pois simular o percurso de uma prova e ter conhecimento das reações do corpo naquele ambiente é de fundamental importância para um bom desempenho.

\chapter{Manual do Produto} % (fold)
\label{cha:manual_do_produto}
 
O trecho a seguir descreve o processo recomendado pela equipe para o uso do produto, assim como os cuidados que devem ser tomados com o manuseio.

\section{Modo de Uso} % (fold)
\label{sec:modo_de_uso}

\begin{enumerate}
	\item Verifique se a aplicação já esta em funcionamento com o auxilio da tela secundaria (notebook/monitor).
	\item Posicione-se sobre a bicicleta. Verifique se a mesma se mantem  estável com o seu peso. Para mais informações sobre os valores para qual este produto foi projetado, verificar \autoref{dados-usuario}.
	\item Imerja-se na realidade virtual com o \gls{rift}.
	\item Após o uso do equipamento, remova o \gls{rift} e espere de 15 a 30 segundos antes de tentar desmontar da bicicleta\footnote{Devido a alta imersão, é recomendado um tempo para se acostumar novamente com a realidade antes de realizar movimentos bruscos.}.
\end{enumerate}

\section{Precauções} % (fold)
\label{sec:precau_es}

\begin{description}
	\item [Não tente desmontar da bicicleta utilizando o gls{rift}] $-$ Ao utilizar o \gls{rift}, a imersão proporcionará uma perca da noção de espaço e posição da realidade. Tentar se movimentar na realidade visualizando o ambiente virtual é extremamente desencorajador.
	\item [Não grite] $-$ A imersão no ambiente virtual causa uma perca de contato com a realidade, lembre-se disto antes de tentar se comunicar com aqueles que te assistem.
	\item [Evite balançar muito na bicicleta ou ficar em pé nela] $-$ Apesar de cálculos terem sido feitos no intuito de manter a bicicleta o mais estável possível ao chão, o produto ainda é passível de tombamento quando exposto a determinados momentos angulares.
	\item [Divirta-se] $-$ Mesmo tendo como objetivo apresentar uma alternativa para a pratica de atividades físicas em ambientes fechados, o projeto busca oferecer também entretenimento ao usuário.
	\item [Mantenha o produto] $-$ Evite reajustar os sensores, a altura do guidão ou do banco. Por ser um protótipo, o produto ainda não oferece um amplo grau de personalização. Tentar configurar o equipamento sem o devido conhecimento do produto pode vir a danifica-lo ou  ao comprometimento dos sinais adquiridos para controle do ambiente.
\end{description}