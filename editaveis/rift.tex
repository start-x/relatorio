\section{Oculus Rift} % (fold)
\label{sec:oculus_rift}

O \textit{Oculus Rift}\cite{oculusVR} é um produto desenvolvido pela Oculus VR\textsuperscript{\textregistered} que foi fundada por Palmer Luckey, um entusiasta em realidade virtual e \textit{nerd} de hardware. A companhia lançou uma campanha no \textit{Kickstarter}\cite{kickstarter}, uma plataforma que permite inventores encontrar patrocinadores, para ajudar a levantar fundos para seu primeiro produto, o \textit{Oculus Rift}, um óculos de imersão virtual bastante inovador para jogos. Com o suporte de gigantes produtoras de jogos eletrônicos como a Valve, Epic Games e Unity, o \textit{Kickstarter} foi o maior sucesso, levantando mais de US\$2,4 milhões em fundos. O time atualmente trabalha fortemente na comercialização do \textit{Oculus Rift}, que promete revolucionar a maneira que as pessoas interagem com conteúdos. 

\subsection{Especificações de Hardware}
O \textit{Oculus Rift} é produzido somente em versões de desenvolvimento, não se sabe ao certo quais serão as especificações de hardware para o produto comercializável. A listagem a seguir apresenta as características de hardware do \textit{Oculus Rift DK1} (primeira versão de desenvolvimento):
\begin{itemize}
	\item Especificações da tela:
		\begin{itemize}
			\item Área visível de 7 polegadas
			\item Resolução total de 1280x800, 640x800 para cada olho
			\item Distancia fixa de 64mm entre os centros das lentes
			\item LCD com frequência de 60Hz
			\item HDMI 1.3+
		\end{itemize}
	\item Especificações dos sensores:
		\begin{itemize}
			\item Ate 1000Hz de taxa de amostragem
			\item Giroscópio de três eixos, para sensorear velocidade angular
			\item Magnetômetro de três eixos, para sensoriais campos magnéticos
			\item Acelerômetro de três eixos, para sensorear a aceleração, incluindo a gravitacional
		\end{itemize}
	\item Conexoes:
		\begin{itemize}
			\item USB 2.0 para transmissão de dados dos sensores
			\item HDMI para transmissão de imagens
			\item Fonte de energia para alimentação do óculos
		\end{itemize}
\end{itemize}

%exemplo de uso do glossário
%O \gls{rift}  é um equipamento de realidade virtual para jogos eletrônicos.
