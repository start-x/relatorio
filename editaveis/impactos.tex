
\chapter[Análise do Impacto Social e Ecônomico do Posicionador de Lente]{Análise do Impacto Social e Ecônomico do Posicionador de Lente}

O produto que está sendo desenvolvido, Posicionador de Lente, é um produto inexistente atualmente no mercado. Com isso, a sua inserção no mercado ocasionará impactos de natureza
social e econômica. Nas seções a seguir será apresentado e discutido cada um deles. É importante relembrar os dois problemas que esse produto procura resolver: limpeza inadequada e dificuldade de utilização. A limpeza inadequada está fortemente relacionada ao impacto social que o produto irá gerar,  pois reduzirá eventuais doenças causadas pela limpeza inadequada, enquanto a dificuldade de utilização está relacionada tanto ao impacto social quanto ao impacto econômico, pois as pessoas irão adquirir o produto porque não conseguem utilizar lentes de contato sem ele, seja qual for a dificuldade.

\section[Impacto Social]{Impacto Social}

\subsection[Redução de doenças causadas pela limpeza inadequada]{Redução de doenças causadas pela limpeza inadequada}

A correta manutenção das lentes de contato é fundamental para se obter
sucesso e manter a continuidade de seu uso. É grande o número de
pacientes que abandonam o uso de suas lentes por problemas que poderiam
ser solucionados com tratamentos relativamente simples ou com uma
orientação mais adequada. O mau uso das lentes, associado à má adaptação,
contaminação, doenças oculares prévias e fatores ambientais, podem
aumentar o número de infecções corneanas através da proliferação de
microorganismos. 

A importância da manutenção das lentes de contato (LC) é fundamental
para se obter sucesso e manter a continuidade de seu uso, já que as
complicações oculares devido à falta de obediência dos usuários em relação
à manutenção e ao período de troca, em conjunto com a ausência de
motivação estão entre as principais causas da desistência do paciente \cite{coral}.

É grande o número de pacientes que abandonam o uso de suas lentes
por causa de problemas que poderiam ser solucionados com tratamentos
relativamente simples \cite{vieira}. Um fator decisivo para que algumas
pessoas desistam de usar suas lentes é a manutenção inadequada ou o uso
incorreto. 

O mau uso das lentes, associado à má adaptação, contaminação, doenças
oculares prévias e fatores ambientais, podem aumentar o número de
infecções corneanas através da proliferação de microorganismos como
bactérias, fungos, parasitas, vírus, uma vez que o próprio uso da lente
altera o mecanismo de defesa do olho \cite{robert}.

As soluções multiuso vieram para facilitar o paciente no
cuidado das LC, uma vez que limpeza, enxágüe e desinfecção
são feitos com o mesmo produto. Suas macromoléculas reduzem
a penetração do desinfetante na córnea, limitando seu
acúmulo na matriz da lente.

Ao contrário do que ocorria há alguns anos, quando os
usuários tinham que limpar suas lentes com inúmeros produtos
diferentes ou até fervê-las, a manutenção tornou-se, atualmente,
mais simples e prática, o que pode fazer o paciente
aderir mais facilmente aos procedimentos \cite{rakow}.

Toda lente, independente de ser rígida ou gelatinosa,
descartável ou convencional, deve, a princípio, passar por
um processo de manutenção que inclua limpeza diária, desinfecção
e retirada dos depósitos de proteínas. Os procedimentos comuns para limpeza de lente são: limpeza do estojo, higienização das mãos, limpeza das lentes, enxague e desinfecção.

A manutenção correta das lentes é essencial na prevenção
de complicações infecciosas, tóxicas e alérgicas, bem como
para o conforto de seu uso. Como já frisamos anteriormente, é
grande o número de pacientes que abandonam as lentes por
complicações causadas por falta de cuidados ou produtos
inapropriados \cite{moreira}.

Assim, um produto como o Posicionador de Lente realizaria de forma automática os procedimentos comuns de limpeza de lente, fazendo com que o usuário interfira minimamente sobre o processo, sendo necessária uma intervenção apenas mensal para manutenção do aparelho. Com isso, o número de desistentes e de possíveis doenças causadas pela limpeza inadequada reduziria.

Outro fato importante é que muitas pessoas ao desistirem de utilizar lentes de contato, deixam de lado também os óculos de grau ou um acompanhamento oftalmológico adequado, fazendo com que as doenças oftalmológicas permaneçam ou se agravam em médio ou longo prazo. 

\subsection[Promoção à Saúde e Qualidade de Vida - Projeto Olhar Brasil]{Promoção à Saúde e Qualidade de Vida - Projeto Olhar Brasil}

São conhecidos os altos percentuais de problemas oftalmológicos que afetam a população 
brasileira e a desigual distribuição dos recursos humanos e financeiros para a sua abordagem. Os 
problemas visuais respondem por grande parcela de evasão e repetência escolar, pelo desajuste 
individual no trabalho, por grandes limitações na qualidade de vida, mesmo quando não se trata 
ainda de cegueira \cite{olhar}.

O SUS dispõe de 2.374 unidades de saúde que realizam consulta oftalmológica. No ano 
de 2005, foram realizadas 7.815.134 consultas oftalmológicas e fornecidos 91.390 óculos. O 
número de oftalmologistas, no Sistema Único de Saúde, totaliza 5.701 profissionais, sendo que a 
maior concentração de médicos oftalmologistas e de unidades de saúde está nas regiões sul e 
sudeste
, com 57\% do total de serviços; 32\% nas regiões norte e nordeste e 11\% na região 
centro-oeste \cite{olhar}.

Evidencia-se a necessidade de realização de novas ações que atendam com maior 
resolutividade à crescente demanda ampliando o acesso da população aos serviços de 
oftalmologia. Com isso, foi criado e implatado o Projeto Olhar Brasil, cujos objetivos são:
\begin{enumerate}
\item Identificar problemas visuais, relacionados a refração, em alunos matriculados na rede 
pública de ensino fundamental (1ª a 8ª série), no programa “Brasil Alfabetizado” do MEC 
e população acima de 60 anos de idade; 
\item  Prestar assistência oftalmológica com fornecimento de óculos nos casos de erro de 
refração; 
\item Otimizar a atuação dos serviços especializados em oftalmologia, ampliando o acesso à
consulta, no âmbito do SUS; 
\item Garantir a referência para serviços especializados nos casos que necessitarem de 
intervenções de Média e Alta Complexidade em Oftalmologia; 
\item Criar um banco de dados com informações do desenvolvimento do Projeto; 
\item Propiciar condições de saúde ocular favorável ao aprendizado do público alvo 
melhorando o rendimento escolar dos estudantes do ensino público fundamental, jovens e 
adultos do programa Brasil Alfabetizado de forma a reduzir as taxas de evasão e 
repetência; 
 \end{enumerate}

Esse projeto atinge cerca de 2 milhões de pessoas. Se uma parceria fosse realizada com o produto Posicionador de Lente o impacto social seria grande. Pois assim como o uso de óculos de grau aumentam a qualidade de vida dessas pessoas, o uso adequado de lentes de contatos iria aumentar ainda mais a auto estima das pessoas: muitas delas não aderem de forma completa ao Projeto Olhar Brasil por não desejarem usar óculos de grau ou os utilizam eventualmente.


\section[Impacto Econômico]{Impacto Econômico}

A dificuldade de utilização de lentes de contato interfere diretamente na venda das mesma. Com a inserção do Posicionador de Lente no mercado, a projeção de vendas de lentes de contato de diversos tipos será crescente. Por exemplo, o público feminino, que é um dos que possuem dificuldade de utilização devido ao comprimento das unhas, seria um grande consumidor tanto do Posicionador de Lente quanto das lentes de contato. Dos dados mais estáveis que foram capturados cerca de 67\% das lentes foram prescritas para mulheres em 2012.