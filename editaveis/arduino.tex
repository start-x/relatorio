\subsection{Arduindo}
\label{sec:arduino}

O Arduino é plataforma \textit{open-source} criada de forma a facilitar o uso da integração de \textit{software} e \textit{hardware}. Foi utilizado o microcontrolador Arduino Mega que é baseado no Atmega 1280. Tem-se um total de 54 pinos digitais de entrada e saída no qual 14 deles tem a possibilidade de serem usados como um sinal PWM (\textit{Pulse Width Modulation}), 16 pinos para entrada de sinais analógicos, 4 portas para comunicação UART (\textit{Universal asynchronous receiver/transmitter}), um oscilador de cristal de 16 MHz, uma conexão USB, uma entrada de alimentação e um botão de \textit{reset}.

O ATmega1280 possui 128 KB de memória \textit{flash} para armazenamento de código (dos quais 4KB são usados pelo \textit{bootloader}), 8 KB de SRAM e 4 KB de EEPROM (que poder ser lidos e escritos com a biblioteca EEPROM). Outra característica importanta é a quantidade de portas de entrada e saída disponíveis, o que permitiu uma maior liberdade ao receber os sinais dos sensores; 

