Este capítulo descreve partes do sistema como um todo, abrangendo ferramentas de controle de estrutura, módulos de interface, hardwares envolvidos e afins. Serão apresentados os desenvolvimentos das partes de:

\begin{itemize}
	\item \nameref{software}
	\begin{itemize}
		\item Controle de infraestrutura
		\item Interface de controle
		\item Visualização de dados
	\end{itemize}
	\item \nameref{eletrônica}
	\begin{itemize}
		\item Aquisição de dados
		\begin{itemize}
			\item Circuito do sistema
			\item Interface de aquisição e disponibilização dos dados (microcontrolador)
		\end{itemize}
	\end{itemize}
	\item \nameref{automotiva}
	\begin{itemize}
		\item Estrutura do produto e seus esforços
	\end{itemize}
	\item \nameref{energia}
	\begin{itemize}
		\item Acoplamento do alternador
		\item Cálculos de eficiência
	\end{itemize}
\end{itemize}


\chapter{Software}
\label{software}

\section{Puppet} % (fold)
\label{sec:puppet}

Para agilizar e evitar os problema com compatibilidades e erros de versões de bibliotecas e outros ativos, foi utilizado a ferramenta \href{http://puppetlabs.com}{Puppet} para auxiliar com este processo administrativo de desenvolvimento do sistema. O Puppet está sendo usado como ferramanta para gerenciamento de configuração do ambiente de desenvolvimento do projeto Bike-X.

\subsection{O que é o Puppet} % (fold)
\label{sub:o_que_o_puppet}

\gls{puppet} é uma ferramenta de gerenciamento de configuração feita em \gls{ruby} que permite que seja definido o estado da infraestrutura de TI (Tecnologia da Informação), com isso, a configuração pode ser replicada em qualquer ambiente, levando em consideração as restrições de implementação da definição do ambiente. Para gerenciamento de alguns servidores ou maquinas, sejam elas físicas ou virtuais, o Puppet automatiza as tarefas administrativas do sistema que normalmente são feitas manualmente, liberando tempo e espaço mental dos administradores de sistemas para trabalhar nos projetos que proporcionam maior valor ao negócio.

O Puppet possui uma sintaxe própria, o mesmo utiliza uma linguagem declarativa, ao invés de uma linguagem imperativa, que é utilizada pela maioria das ferramentas similares, como por exemplo o \gls{chef}. A linguagem imperativa é um conceito baseado em estados, definidos por variáveis, e ações que são manipuladoras de estado, procedimentos. Pelo fato de permitir o uso de procedimentos como estruturação, também é conhecido como procedural. A linguagem declarativa, ao contrário da linguagem imperativa que informa ao computador "como" as instruções devem ser executadas, preocupa-se em apenas dizer ao computador "o que" precisa ser feito, cabendo ao computador decidir qual a melhor solução para essa solicitação.
 

\subsection{Integrando o Puppet ao projeto} % (fold)
\label{sub:integrando_o_puppet_ao_projeto}

Para a utilização do Puppet dentro do projeto foi desenvolvido um módulo puppet específico, chamado \href{https://github.com/start-x/startx-src/tree/master/puppet}{bike-x}. Nesse módulo é definido nome e versão de pacotes a serem instalados, é definido links simbólicos, criação de arquivos de configuração para o \gls{rift}. O módulo em questão foi testado e homologado para algumas distribuições do sistema operacional Linux:

\begin{itemize}
\item Ubuntu 13.10
\item Ubuntu 14.04
\item Mint 17
\item Debian Wheezy
\item Debian Sid
\end{itemize}

Todas as distribuições apresentadas são "Debian like" e o módulo está totalmente modularizado para a adição de qualquer nova distribuição necessária, entretanto, as apresentadas acima atende 100\% a equipe de desenvolvimento do projeto. A lista de pacotes Debian necessários para o projeto estão listados a seguir:

\begin{multicols}{2}
\begin{itemize}
\item python2.7
\item python2.7-dev
\item python-serial
\item doxygen
\item graphviz
\item dots
\item binutils-msp430
\item gcc-msp430
\item msp430-libc
\item msp430mcu
\item mspdebug
\item libxext-dev
\item mesa-common-dev
\item freeglut3-dev
\item libxinerama-dev
\item libxrandr-dev
\item libxxf86vm-dev
\item swig
\end{itemize}
\end{multicols}

A seguir lista de pacotes Python utilizados:

\begin{itemize}
\item pyserial
\item mock
\end{itemize}

Após a execução do módulo Puppet desenvolvido, todos os pacotes listados acima estarão instalados no sistema. Para facilitar a execução do módulo Puppet para todos da equipe, foi desenvolvido um Shell script chamado "setup\_development\_environment.sh" onde é instalado o Puppet em si e os módulos que são dependência, que são o \textit{pip} e \textit{stdlib}, e executado o módulo Puppet bike-x.


