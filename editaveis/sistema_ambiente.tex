Este capítulo descreve partes do sistema como um todo, abrangendo ferramentas de controle de estrutura, módulos de interface, hardwares envolvidos e afins. Serão apresentados os desenvolvimentos das partes de:

\begin{itemize}
	\item \nameref{software}
	\begin{itemize}
		\item Controle de infraestrutura
		\item Interface de controle
		\item Visualização de dados
	\end{itemize}
	\item \nameref{eletronica}
	\begin{itemize}
		\item Aquisição de dados
		\begin{itemize}
			\item Circuito do sistema
			\item Interface de aquisição e disponibilização dos dados (microcontrolador)
		\end{itemize}
	\end{itemize}
	\item \nameref{automotiva}
	\begin{itemize}
		\item Estrutura do produto e seus esforços
	\end{itemize}
	\item \nameref{energia}
	\begin{itemize}
		\item Acoplamento do alternador
		\item Cálculos de eficiência
	\end{itemize}
\end{itemize}


\chapter{Software}
\label{software}

\section{Puppet} % (fold)
\label{sec:puppet}

\subsection{O que é o Puppet} % (fold)
\label{sub:o_que_o_puppet}

\subsection{Integrando o puppet ao projeto} % (fold)
\label{sub:integrando_o_puppet_ao_projeto}

% subsection integrando_o_puppet_ao_projeto (end)

\section{Interface Python} % (fold)
\label{sec:interface_python}

A interface \gls{python} simplifica a comunicação com o microcontrolador, possibilitando o \textit{parse} entre o modulo principal (BikeX \ref{sec:sistema_bikex}) e o \gls{msp}.

A interface faz de uso da biblioteca \href{http://pyserial.sourceforge.net/pyserial.html}{Pyserial} para manter a comunicação com o microcontrolador. Devido as inúmeras possibilidades de conflitos existentes de caracteres e velocidade de comunicação existentes na comunicação \gls{rs232}, os desenvolvedores da \textit{Pyserial} construíram a classe \textit{serial.tools.miniterm} na qual simula um terminal de comunicação como exemplo de uso da biblioteca. O grupo construiu então uma classe que herda da \textit{serial.tools.miniterm}, simplificando assim a comunicação e incrementando a estabilidade de comunicação. Esta ação gera a dependência de que a versão da \textit{Pyserial} necessita ser 2.7 ou superior.

\subsection{Visão do BikeX} % (fold)
\label{sub:vis_o_do_bikex}

Do ponto de vista do BikeX a aplicação Python estará rodando sempre em \textit{background} esperando um sinal para a realização de alguma tarefa. A depender do sinal recebido, será realizado uma leitura do estado dos sensores ou o envio do valor de posicionamento do freio.

% Por ser um programa assíncrono, a aplicação passará grande parte do tempo ociosa.
A tomada de uso de sinais para acordar o processo possibilita que o mesmo se mantenha em descanço durante todo o período em que não há requisição de dados. Como resultado do ponto de vista da arquitetura que suportado software, não há uma sobrecarga de processamento, evitando o \textit{overhead} de requisições desnecessárias e possibilitando assim que o processamento possa ficar focado onde realmente há uma grande demanda de CPU e GPU, que é a interface gráfica da aplicação.

%../startx-src/device/scripts/main.py

\begin{lstlisting}[language=Python,caption={Trecho do código principal}]

def main():
    pid_bikex = sys.argv[1:2]
    print "Father: %s" % pid_bikex
    print "Mine: %d " % getpid()
    print SIG1,SIG2,SIG3
    msp430.curb = sensor.Break(msp430.serial,0)
    msp430.passives = sensor.Passives(msp430.serial,0)

    signal(SIG1, write_file)
    signal(SIG3, read_file)
    signal(SIGINT, safe_quit)
    signal(SIGQUIT, safe_quit)
    signal(SIGABRT, safe_quit)
    while True:
        sleep(0.01)

\end{lstlisting}


\subsection{Visão do MSP430} % (fold)
\label{sub:vis_o_do_msp430}

Do ponto de vista do MSP430 a aplicação Python estará sempre em comunicação ativa com o MSP430, já que a porta serial será aberta assim que o sistema for iniciado e só fechará quando programa vier a fechar.
