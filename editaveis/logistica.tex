\chapter[Organização do Trabalho]{Organização do Trabalho}

Para a realização do projeto de forma eficiente e organizada, dividiu-se inicialmente o grupo em quatro subgrupos, cada um destes representando uma das 
engenharias (automotiva, eletrônica, energia e software), e cada subgrupo tendo um representante. No decorrer do projeto, de acordo com as demandas, os 
integrantes dos subgrupos deverão ser permutados. 

Haverá em média cinco reuniões semanais com duração de duas horas entre os integrantes do projeto, sendo três delas durante as aulas e duas extra classe.
O grupo decidiu utilizar uma abordagem ágil de gerenciamento de projeto, tendo em vista que a mesma funciona bem com times pequenos, concentrando ao máximo
o esforço do time em agregar valor ao produto proposto. Esse tipo de abordagem irá favorecer a integração do grupo de trabalho, objetivo principal da disciplina.

As decisões importantes a serem tomadas, como a definição do tema do projeto, as divisões e os principais resultados esperados, são feitas por 
todos os componentes do grupo durante os horários de reunião. Além das tomadas de decisões, as reuniões serão aproveitadas para cada subgrupo se reunir, 
trabalhar em sua determinada área, apresentar e discutir seus resultados obtidos para os demais subgrupos e, quando necessário, apresentar suas principais 
dificuldades e questionamentos para os professores da disciplina. Essas horas também serão úteis para que as tarefas em que é necessário mais de um subgrupo 
para sua realização sejam cumpridas através da reunião entre os mesmos para coletar as informações necessárias e discutir os melhores métodos e soluções para 
essas tarefas.

Foi estimada uma média de quatro horas semanais de trabalho além das seis horas de aula para cada componente do grupo, a fim de concluir as tarefas e metas 
propostas para cada um desses. Essas horas são utilizadas em sua maioria para pesquisas, testes, simulações e atualizações do relatório. 

Segundo \cite{XPxRUP2006}, metodologias ágeis também dividem o desenvolvimento do software em iterações, buscando redução de riscos ao projeto. Ao final de cada iteração, uma versão (release) funcional do produto, embora restrita em funcionalidades, é liberada ao cliente. As metodologias ágeis destacam aspectos humanos no desenvolvimento do projeto, promovendo interação na equipe de desenvolvimento e o relacionamento de cooperação com o cliente. Comunicação face-a-face é preferida à documentação compreensiva.

Com o objetivo de aperfeiçoar a integração entre os componentes dos grupos e para que cada um possa acompanhar o andamento do projeto são utilizadas algumas 
ferramentas e práticas ágeis, como \textit{software} de gerenciamento de projeto e \textit{daily meetings}. Assim, de modo que cada componente e/ou subgrupo 
possa acompanhar o que os outros estão fazendo no projeto está sendo utilizado: a ferramenta livre \href{http://lappis.unb.br/redm}{Redmine}, onde são apresentadas as tarefas, seus andamentos e o 
responsável por cada uma delas em um quadro \textit{kanban}; e os encontros diários, onde todos dizem o que foi feito, o que está sendo feito, as dificuldades
e o que será feito, possibilitando com que os principais problemas e dificuldades sejam detectados e solucionados por todos em conjunto. Para o agrupamento 
dos dados e pesquisas coletadas, além dos testes e resultados gerados e atualizações do relatório, é utilizada a ferramenta Google Docs.

Está sendo utilizada a ferramenta Git para realizar o controle de versão tanto dos códigos fonte gerados quanto dos documentos e apresentações, e como Source
Forge está sendo utilizado o GitHub. Sendo o Git uma ferramenta livre e o GitHub gratuito.

Com essa maneira de organizar o tempo, as tarefas e as equipes, espera-se que o andamento do projeto seja satisfatório, integrando as engenharias através do 
trabalho entre os subgrupos de maneira eficiente. Além disso, objetiva-se o melhor aproveitamento possível das horas disponíveis e determinadas para a realização 
do projeto por todos os componentes, de modo que a divisão de trabalho seja equilibrada ao longo do projeto, o que pode ser observado e analisado através das 
ferramentas utilizadas para o controle e divisão de tarefas.

