\newglossaryentry{PGR}
{
	name = Plano de Gerenciamento de Requisitos,
	text = Pla\-no de Ge\-ren\-ci\-a\-men\-to de Re\-qui\-si\-tos,
	description={Projeto que tem como objetivo principal de apresentar como produto um documento 
contendo  a  descrição  detalhada da etapa inicial do projeto de um Sistema de Pontos  de Função 
para atender às necessidades no contexto da Empresa CÁRTAMO S.A.}
}


\newglossaryentry{USU}
{
	name = Usu\'ario,
	text = Usu\'ario,
	description={O usuário pode ser qualquer pessoa ou coisa que se comunica com a aplicação ou interage com ela a qualquer momento}
}


\newglossaryentry{VIS}
{
	name = Vis\~ao,
	text = Vis\~ao,
	description={Documento de visão gerado pelo analista de sistema}
}


\newglossaryentry{NEC}
{
	name = Necessidade do Interessado,
	text = Necessidade do Interessado,
	description={Necessidades apontadas dos stakeholders}
}

\newglossaryentry{UC}
{
	name = Caso de Uso,
	text = Caso de Uso,
	description={Documento gerado a partir dos stakeholders, mas é de autoria e propriedade do analista de sistema. Conta uma história (descrição geral das tarefas ou interações) sobre como o usuário interage com o sistema dado as circunstâncias},
	plural=Casos de Uso,
}



\newglossaryentry{EUC}
{
	name = Especifica\c c\~ao de Casos de Uso,
	text = Especifica\c c\~ao de Casos de Uso,
	description={Documento gerado a partir dos stakeholders, mas é de autoria e propriedade do analista de sistema}
}

\newglossaryentry{ATR}
{
	name = Ator,
	text = Ator,
	description={Pessoas, dispositivos ou mesmo sistemas externos que usam o produto no contexto da função e comportamento},
	plural=Atores,
}


\newglossaryentry{RF}
{
	name = Requisito Funcional,
	text = Requisito Funcional,
	description={São as declarações de serviços que o sistema deve fornecer, como o sistema de reagir entradas específicas e como o sistema deve se comportar em determinadas situações}
}

\newglossaryentry{NF}
{
	name = Requisito N\~ao Funcional,
	text = Requisito N\~ao Funcional,
	description={São restrições sobre os serviços ou as funções oferecidos pelo sistema. Eles incluem restrições de \textit{timing}, restrições sobre o processo de desenvolvimento e padrões}
}

\newglossaryentry{AH}
{
	name = Ambiente de Homologa\c c\~ao,
	description={O ambiente de homologação é um ambiente onde o
cliente deverá testar as funcionalidades do sistema que serão posteriormente colocadas
em produção ou refeitas, caso não ocorra à aprovação por parte deste cliente}
}

\newglossaryentry{AD}
{
	name = Ambiente de desenvolvimento,
	description={Um ambiente de desenvolvimento contém o que é
necessário para uma equipe construir e implementar sistemas com alto uso de software
(nos quais o software é um elemento essencial e indispensável)}
}