\newglossaryentry{rift}
{
	name = Oculus Rift,
	text = \textit{Oculus Rift},
	description={Equipamento de realidade virtual para jogos eletrônicos desenvolvido pela Oculus VR e adquirido pelo \href{https://www.facebook.com/facebook}{Facebook In} em 25 de março de 2014}
}

\newglossaryentry{msp}
{
	name = MSP430,
	text = \textit{MSP430},
	description={Microcontrolador de baixa potência da Texas Instruments (TI). A versão utilizada neste projeto é a MSP430G2553, disponibilizada junto ao kit \href{http://www.ti.com/tool/msp-exp430g2}{Launchpad}}
}


\newglossaryentry{arduino}
{
	name = Arduino,
	text = \textit{Arduino},
	description={Arduino, ou Arduíno é uma plataforma de prototipagem eletrônica de hardware livre e de placa única, projetada com um microcontrolador Atmel AVR com suporte de entrada/saída embutido. Usa uma linguagem de programação padrão,de origem em $Wiring$, sendo essencialmente $C/C++$.}
}

\newglossaryentry{rs232}
{
	name = UART (RS232),
	text = \textit{UART},
	description={Padrão de protocolo para troca serial de dados binários entre um DTE (terminal de dados) e um DCE(comunicador de dados). Também é conhecido por \textit{EIA RS-232C} ou \textit{V.24)}}
}

\newglossaryentry{python}
{
	name = Python,
	text = \textit{Python},
	description={Python é uma linguagem de programação de alto nível , interpretada, imperativa, orientada a objetos, funcional, de tipagem dinâmica e forte. Foi lançada por Guido van Rossum em 1991\cite{python_history}. Atualmente possui um modelo de desenvolvimento comunitário, aberto e gerenciado pela organização sem fins lucrativos Python Software Foundation}
}

\newglossaryentry{puppet}
{
	name = Puppet,
	text = \textit{Puppet},
	description={Puppet é um sistema de gerenciamento de configuração que permite que que seja definido o estado da infraestrutura de TI. A ferramenta open-source é mantida pela Puppet Labs}
}

\newglossaryentry{git}
{
	name = Git,
	text = \textit{Git},
	description={É um sistema de controle e versionamento de código open-source desenvolvido por Linus Torvalds e Junio Hamano sob licença GNU GPL2. Permite o desenvolvimento distribuído, autenticação criptográfica do histórico, estratégias de mescla (merge) conectáveis dentre outras funcionalidades que agilizam a produção e garantem integração do desenvolvimento da equipe com o projeto, fazendo dele uma das gerramentas mais utilizadas atualmente para versionamento de código}
}

\newglossaryentry{cave}
{
	name = CAVE,
	text = \textit{CAVE},
	description={Caverna digital ou CAVE (Cave Automatic Virtual Environment) é um ambiente onde são projetados imagens em suas paredes (e algumas vezes chão) no intuito de explorar e interagir com objetos, pessoas virtuais e outros para ter um ambiente virtual, desta forma megulhando em um mundo virtual}
}

\newglossaryentry{chef}
{
	name = Chef,
	text = \textit{Chef},
	description={Chef é um sistema de gerenciamento de configuração que permite que que seja definido o estado da infraestrutura de TI. Usa Ruby e  domain-specific language (DSL) realizar as \textit{receitas} no sistema}
}

\newglossaryentry{ruby}
{
	name = Ruby,
	text = \textit{Ruby},
	description={Ruby é uma linguagem de programação dinâmica, refletiva, orientada à objetos de proposito de uso geral. Ela foi designada e desenvolvida nos anos 90 por Yukihiro "Matz" Matsumoto no Japan}
}

\newglossaryentry{unity}
{
	name = Unity 3D,
	text = \textit{Unity},
	description={Também identificado como Unity 3D para evitar equívocos de plataformas, o Unity é um ecossistema de desenvolvimento de jogos, com mecanismos de renderização, ferramentas de fluxos de trabalho rápido, com suporte para publicação em multiplataforma e bibliotecas prontas para uso.}
}
