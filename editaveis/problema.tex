\chapter[Problema]{Problema}

O estilo de vida do ser humano mudou drasticamente ao longo dos últimos séculos. Atividades que antes eram realizadas por pessoas passaram a ser executadas 
por máquinas, e sistemas eletrônicos, exigindo cada vez menos do corpo humano. Esta diminuição em atividades físicas tem levado as pessoas a tomar um estilo 
de vida mais sedentário, provocando um aumento nos índices de doenças crônicas como obesidade, diabetes, hipertensão e uma série de outras doenças. Desta forma, 
as organizações de saúde recomendam uma alimentação balanceada e principalmente a prática de exercícios físicos para combater os problemas causados pelo 
sedentarismo.

Dentre as diversas formas de atividade física, o ciclismo se destaca por ser uma atividade simples e agradável ao ciclista. O ciclismo permite ao praticante 
exercitar vários músculos do corpo e trabalhar a coordenação motora. Além disso, o contato com o ambiente torna esta prática mais prazerosa ao usuário. No 
entanto, existem diversos fatores que dificultam a prática desta atividade nas grandes cidades brasileiras.
A primeira das dificuldades enfrentadas por ciclistas é o acesso a espaços apropriados para a prática do ciclismo. A ausência de ciclovias obriga ao ciclistas 
a utilizar as ruas e a dividir espaço com motoristas, que geralmente não aceitam dividir o espaço pelo fato dos ciclistas trafegarem em velocidades menores que 
a dos carros, podendo causar acidentes. Mesmo quando há ciclovias, os ciclistas tem de enfrentar as avenidas urbanas para chegar ao local, novamente interagindo 
com motoristas.

Outros fatores que tornam a prática do ciclismo mais difícil envolvem questões relacionadas a falta de estrutura urbana. Muitas cidades brasileiras não são 
planejadas para o ciclismo, mesmo como forma de locomoção, pois quando há ciclovias, elas são geralmente descontínuas. 

Em Pelotas, Rio Grande do Sul, Brasil, \cite{barros2003}, comparando informações de boletins de ocorrência e atendimentos no pronto-socorro durante dois anos, encontraram 33,0\% de sub-registros relativos aos acidentes com lesão corporal envolvendo ciclistas.

Buracos e falta de recapeamento nas ciclovias também podem acarretar em acidentes e causar lesões aos ciclistas. Os ciclistas mais regulares afirmam também que a falta de bicicletários os obrigam a improvisar formas de fixar seu veículo. Outro fator que contribui para as dificuldades enfrentadas por ciclistas é a falta de segurança e de estrutura em espaços públicos, onde a falta de iluminação ou mesmo as abordagens de assaltantes tornam a prática mais arriscada.

Para as pessoas que não praticam atividades físicas, a principal justificativa é a falta de tempo com relação a suas atividades diárias. Em especial para o 
ciclismo, de fato, é necessário um gasto de tempo até a chegada em uma área apropriada para uma circulação mais tranquila de bicicletas. Como alternativa, as 
bicicletas ergométricas que em sua maioria estão presentes em academias são um incentivo a prática do ciclismo e devido a sua estrutura mecânica, tendem a ser 
mais confortáveis para o praticante.
Outra característica destas bicicletas é que pelo fato de serem fixas, oferecem menos riscos de lesões causadas por quedas. Além disso, o stress causado pelo 
trânsito em avenidas movimentadas não existe para este caso. Apesar de permitir simular os movimentos da pedalada de uma bicicleta comum, a bicicleta ergométrica 
tem a desvantagem de não oferecer estímulo do ambiente ao usuário. Com isso, exercitar-se em uma bicicleta ergométrica se torna uma atividade monótona ao 
praticante.


