\begin{anexosenv}

\partanexos

\chapter{ Comandos utilizados para simulação no Ansys}

Abaixo os comandos que foram utilizados para a simulação da Bike 3D no Ansys.

!Comandos para título da simulação - Bike 3D Start-X

/title,Bike Start-X

/prep7 ! Iniciando pre-processor

! Definindo alguns parametros

x1 = 500 ! Parametros desnecessarios, usados a seguir para facilicar os comandos.		

x2 = 825 

y1 = 325 

y2 = 400 

z1 = 50 

! Definindo os Keypoints 

K,1, 0,y1, 0 ! k,numero do key-point,coordenada-x,coordenada-y,coordenada-z

K,2, 0,y2, 0 

K,3,x1,y2, 0 

K,4,x1, 0, 0 

K,5,x2, 0, z1 

K,6,x2, 0,-z1 

! Definindo as Linhas, ligando os Keypoints 

L,1,2 ! l,keypoint1,keypoint2 

L,2,3 

L,3,4 

L,4,1 

L,4,6 

L,4,5 

L,3,5 ! Estas duas ultimas linhas, são para o suporte traseiro da bicicleta

L,3,6 

! Definindo o tipo de elemento 

ET,1,pipe16 

KEYOPT,1,6,1 

! Definindo as constantes reais (Caracteristicas geometricas do perfil da bicicleta)

R,1,25,2 ! r,numero real,diametro externo,espessura da parede

R,2,12,1 ! segundo conjunto de constantes reais, para o suporte traseiro

! Definindo as propriedades do material

MP,EX,1,70000 ! mp,modulo de Yong,numero do material,valor

MP,PRXY,1,0.33 ! mp,coeficiente de Poisson's,numero do material,valor (Aluminio)

! Difinindo o número de elementos de cada linha, para a mesma ser dividida

LESIZE,ALL,20 ! (lesize)Numero de linhas(todas as linhas),o tamanho do elemento

! Line Meshing 

REAL,1 ! ativando as propriedades reais (1) 

LMESH,1,6,1 ! malhar as linhas, que possuem tais propriedades

! malhar linhas de 1 a 6, passo igual a 1

REAL,2 ! ativando as propriedades reais (2) 

LMESH,7,8 ! malhar suporte traseiro

FINISH ! finalizando o processo

/SOLU  

ANTYPE,0 ! Tipo de analise,estatica 

! Definindo restriçoes aplicadas nos keypoint

DK,1,UX,0,,,UY,UZ ! dk,keypoint,direção,dislocamento,,,direção,direção

DK,5,UY,0,,,UZ 

DK,6,UY,0,,,UZ 

! Definindo forças aplicadas no Keypoints 

FK,3,FY,-1000 !fk,keypoint,direção,força 

SOLVE ! Resolver o problema

FINISH ! Finalizando a solução

SAVE ! Salvando o solução

/post1

PRRSOL,	! lista reações de apoios 


%\chapter{Segundo Anexo}

%Texto do segundo anexo.

\end{anexosenv}

