
\chapter[Restrições e Trabalhos Futuros]{Restrições e Trabalhos Futuros}


Com relação ao algoritmo do processamento de imagens, ele apresentou, dentro das condições de contorno especificadas, um funcionamento adequado para o produto. Porém, em seu estado final, exigiu uma demanda excessiva de memória por parte do hardware utilizado. Portanto, para trabalhos futuros pretende-se otimizar o algoritmo de detecção para que consuma menos memória. Uma possibilidade é substituir a Linguagem Python por outra que exija menos esforço do hardware.

Com relação ao requisito não funcional de salubridade e a qualidade do produto podem ser melhorados utilizando materiais diferentes da madeira para construção do produto. A madeira foi utilizada, pois fizemos quatro tentativas de impressão nas impressoras 3D da Faculdade UnB Gama, mas não houve êxito, e não conseguimos recurso material diferente para construir o produto no prazo da disciplina Projeto Integrador 2. Portanto, para trabalhos futuros a madeira poderia ser substituída por borracha ou algum material mais apropriado.

Com relação ao site de divulgação, caso o produto entre em produção, para trabalhos futuros seria colocar o site em um servidor web e testar o processo de compra do produto.

Com relação a estrutura do produto, para trabalhos futuros seria reduzir a vibração e ruído do sistema e otimizá-lo.

Com relação ao sitema de armazemanento e limpeza, para trabalhos futuros seria melhorar o design da caixa de armazenamento.

E com relação a atuação e controle do sistema, para trabalhos do futuro seria construir placas definitivas do produto que eliminem a placa de desenvolvimento do arduíno e explorar o tratamento de erros no algoritmo de controle. 