\definecolor{blue}{RGB}{41,5,195}
\makeatletter
\hypersetup{
     	%pagebackref=true,
		pdftitle={\@title}, 
		pdfauthor={\@author},
    	pdfsubject={Bike-X},
	    pdfcreator={LaTeX with abnTeX2},
		pdfkeywords={Rift, }{Unity 3D, }{abntex, }{Sensores Fisiol\'ogicos, }{Ambiente Virtual}, 
		colorlinks=true,       		% false: boxed links; true: colored links
    	linkcolor=blue,          	% color of internal links
    	citecolor=blue,        		% color of links to bibliography
    	filecolor=magenta,      		% color of file links
		urlcolor=blue,
		bookmarksdepth=4
}
\makeatother
\setlength{\parindent}{1.3cm}
\setlength{\parskip}{0.2cm}  
\makeindex

\usetikzlibrary{chains}
\usetikzlibrary{shapes,arrows,shadows}
\usetikzlibrary{mindmap,trees}
\usetikzlibrary{calc,decorations.pathmorphing,patterns}
\pgfdeclaredecoration{penciline}{initial}{
    \state{initial}[width=+\pgfdecoratedinputsegmentremainingdistance,
    auto corner on length=1mm,]{
        \pgfpathcurveto%
        {% From
            \pgfqpoint{\pgfdecoratedinputsegmentremainingdistance}
                      {\pgfdecorationsegmentamplitude}
        }
        {%  Control 1
        \pgfmathrand
        \pgfpointadd{\pgfqpoint{\pgfdecoratedinputsegmentremainingdistance}{0pt}}
                    {\pgfqpoint{-\pgfdecorationsegmentaspect
                     \pgfdecoratedinputsegmentremainingdistance}%
                               {\pgfmathresult\pgfdecorationsegmentamplitude}
                    }
        }
        {%TO 
        \pgfpointadd{\pgfpointdecoratedinputsegmentlast}{\pgfpoint{1pt}{1pt}}
        }
    }
    \state{final}{}
}


% \usepackage{tikzgraphicx}

\tikzstyle{cloud} = [draw, ellipse,fill=blue!20, node distance=3cm,
    minimum height=3em]
\tikzstyle{estado} = [draw, ellipse,fill=blue!20, node distance=3cm,align=center]
\tikzstyle{teste} = [draw, diamond,fill=green!20, node distance=3cm,align=left]
%    minimum height=2em]
\tikzstyle{ciclo} =[draw, rectangle,fill=blue!20, node distance=3cm,align=center,decorate,thick,minimum height=3em]
\tikzstyle{phanton} = []   
\tikzstyle{line} = [->,right] %[draw, -latex']
\tikzstyle{retorno} = [loop above]
\tikzstyle{arrow} = [bend left,->]
\tikzset{
    %Define standard arrow tip
    >=stealth',
    %Define style for boxes
    punkt/.style={
           rectangle,
           rounded corners,
           draw=black, very thick,
           text width=6.5em,
           minimum height=2em,
           text centered},
    % Define arrow style
    pil/.style={
           ->,
           thick,
           shorten <=2pt,
           shorten >=2pt,}
}


%--------------------------------------------
%Definições para código com fundo listrado
\newcommand\realnumberstyle[1]{#1}
\makeatletter
\newcommand{\zebra}[3]{%
    {\realnumberstyle{#3}}%
    \begingroup
    \lst@basicstyle
    \ifodd\value{lstnumber}%
        \color{#1}%
    \else
        \color{#2}%
    \fi
        \rlap{
        \color@block{0.94\textwidth}{\ht\strutbox}{\dp\strutbox}%
        \hspace*{\lst@numbersep}%
        }%
    \endgroup
}
\makeatother

\lstset{%
language=C++,           %linguagem
numbers=left,           %posição dos números
stepnumber=1,           %frequencia de aparição dos números
numbersep=5pt,
numberstyle=\zebra{gray!15}{white!35},
basewidth={0.6em,0.45em},
fontadjust=true,
mathescape=true,
tabsize=4,
commentstyle=\color{blue},
literate={á}{{\'a}}1 {à}{{\`a}}1 {ã}{{\~a}}1 {é}{{\'e}}1 {É}{{\'E}}1 {ê}{{\^e}}1 {õ}{{\~o}}1 {í}{{\'i}}1 {ó}{{\'o}}1 {ú}{{\'u}}1 {ç}{{\c c}}1 {³}{{$^3$}}1 {Ω}{{$\Omega$}}1,
breaklines=true,
showstringspaces=false,
stringstyle=\color{cyan},
basicstyle=\small\ttfamily}

\lstset{%
language=Python,           %linguagem
numbers=left,           %posição dos números
stepnumber=1,           %frequencia de aparição dos números
numbersep=5pt,
numberstyle=\zebra{gray!15}{white!35},
basewidth={0.6em,0.45em},
fontadjust=true,
mathescape=true,
tabsize=4,
commentstyle=\color{blue},
literate={á}{{\'a}}1 {à}{{\`a}}1 {ã}{{\~a}}1 {é}{{\'e}}1 {É}{{\'E}}1 {ê}{{\^e}}1 {õ}{{\~o}}1 {í}{{\'i}}1 {ó}{{\'o}}1 {ú}{{\'u}}1 {ç}{{\c c}}1 {³}{{$^3$}}1 {Ω}{{$\Omega$}}1,
breaklines=true,
showstringspaces=false,
stringstyle=\color{cyan},
xleftmargin=\leftmargini,
columns=flexible,
basicstyle=\small\ttfamily}
