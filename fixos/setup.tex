\definecolor{blue}{RGB}{41,5,195}
\makeatletter
\hypersetup{
     	%pagebackref=true,
		pdftitle={\@title}, 
		pdfauthor={Start-X},
    	pdfsubject={\imprimirpreambulo},
	    % pdfcreator={LaTeX with abnTeX2},
		pdfkeywords={abnt, latex, abntex, abntex2, trabalho acadêmico}, 
		colorlinks=true,       		% false: boxed links; true: colored links
    	linkcolor=black,          	% color of internal links
    	citecolor=black,        		% color of links to bibliography
    	filecolor=magenta,      		% color of file links
		urlcolor=blue,
		bookmarksdepth=4
}
\makeatother
\setlength{\parindent}{1.3cm}
\setlength{\parskip}{0.2cm}  
\makeindex

%Definições para código com fundo listrado
\newcommand\realnumberstyle[1]{#1}
\makeatletter
\newcommand{\zebra}[3]{%
    {\realnumberstyle{#3}}%
    \begingroup
    \lst@basicstyle
    \ifodd\value{lstnumber}%
        \color{#1}%
    \else
        \color{#2}%
    \fi
        \rlap{\hspace*{\lst@numbersep}%
        \color@block{\linewidth}{\ht\strutbox}{\dp\strutbox}%
        }%
    \endgroup
}
\makeatother

\lstset{%
language=C++,						%linguagem
numbers=left,						%posição dos números
stepnumber=1,						%frequencia de aparição dos números
numbersep=5pt,
numberstyle=\zebra{gray!15}{white!35},
basewidth={0.6em,0.45em},
fontadjust=true,
mathescape=true,
tabsize=4,
commentstyle=\color{blue},
literate={á}{{\'a}}1 {à}{{\`a}}1 {ã}{{\~a}}1 {é}{{\'e}}1 {É}{{\'E}}1 {ê}{{\^e}}1 {õ}{{\~o}}1 {í}{{\'i}}1 {ó}{{\'o}}1 {ú}{{\'u}}1 {ç}{{\c c}}1 {³}{{$^3$}}1 {Ω}{{$\Omega$}}1,
breaklines=true,
showstringspaces=false,
stringstyle=\color{cyan},
basicstyle=\small\ttfamily}